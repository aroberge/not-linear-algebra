\documentclass{article}
\usepackage{tikz}
% % %  default French language settings
\usepackage[utf8]{inputenc}	% enable input of accented letters
\usepackage[T1]{fontenc}		% related to above
\usepackage[french]{babel}		% French language support
\usepackage{translator}		% same
	\uselanguage{French}
	\languagepath{French}

\usepackage{amsmath}
\usepackage{amssymb} 

\begin{document}
\section*{Revue}

Que signifie une équation de la forme $\mathbf{A}\mathbf{x} = \mathbf{b}$?
\vfill

\noindent\textbf{\fbox{1.} C'est une façon d'écrire un système d'équations linéaires}
\[
\begin{matrix}
a_{11} x_1 &+& a_{12} x_2 &=& b_1 \\
a_{21} x_1 &+& a_{22} x_2 &=& b_2 
\end{matrix}
\]
\vfill

\noindent\textbf{\fbox{2.} C'est une façon de représenter
une combinaison linéaires de vecteurs.}
Par exemple, avec $\mathbf{A} = [\mathbf{a}_1 \, \mathbf{a}_2]$
\[
 \mathbf{A}\mathbf{x} = \mathbf{b} \qquad\Rightarrow
\qquad
x_1 \mathbf{a}_1 + x_2 \mathbf{a}_2 = \mathbf{b} 
\]
\vfill

\noindent\textbf{\fbox{3.} C'est une façon de représenter
une transformation linéaire.}
\bigskip

Par exemple, soit $\mathbf{b}\in \mathbb{R}^2$ et
$\mathbf{x}\in \mathbb{R}^3$ alors on peut avoir
\[
\mathbf{A}_{2 \times 3} \mathbf{x}_{3\times 1} =
\mathbf{b}_{2\times 1}
\]
\vfill

\noindent\textbf{\fbox{4.} C'est également une façon
de représenter un changement de base.}
\bigskip

Ceci est nouveau.
\vfill

\newpage

\section*{Changement de base}
Nous commençons avec un vecteur $\mathbf{v}$ dans
le plan cartésien. 

\begin{tikzpicture}
\clip (-2,-2) rectangle (10cm,6cm);
\draw[style=help lines,dashed] (-14,-14) grid[step=1cm] (14,14);
\draw[thick,red,->]  (0,0)--(1,0) node[anchor=south]{$\mathbf{e}_1$};
\draw[thick,red,->] (0,0)--(0,1) node[anchor=east]{$\mathbf{e}_2$};
\draw[thick,black,->](0,0)--(7,4) node[anchor=south]{$\mathbf{v}$};
\end{tikzpicture}

Tout vecteur dans ce plan
peut être écrit comme une combinaison linéaire
de deux vecteurs de la base habituelle, également appelée base canonique.
On peut vérifier que $\mathbf{v} = 7\mathbf{e}_1
+ 4 \mathbf{e}_2$

\begin{tikzpicture}
\clip (-2,-2) rectangle (10cm,6cm);
\draw[style=help lines,dashed] (-14,-14) grid[step=1cm] (14,14);
\draw[thick,red,->]  (0,0)--(7,0) node[anchor=north]{$7\mathbf{e}_1$};
\draw[thick,red,->] (7,0)--(7,4) node[anchor=west]{$4\mathbf{e}_2$};
\draw[thick,black,->](0,0)--(7,4) node[anchor=south]{$\mathbf{v}$};
\end{tikzpicture}
\newpage
Introduisons un système de coordonnées différent

\begin{tikzpicture}
\clip (-2,-2) rectangle (10cm,6cm);
\pgftransformcm{1}{-0.5}{1}{1}{\pgfpoint{0cm}{0cm}};
\draw[style=help lines] (-14,-14) grid[step=1cm] (14,14);
\draw[thick,blue,->]  (0,0)--(1,0) node[anchor=south]{$\mathbf{b}_1$};
\draw[thick,blue,->] (0,0)--(0,1) node[anchor=east]{$\mathbf{b}_2$};
\draw[thick,black,->](0,0)--(2,5) node[anchor=south]{$\mathbf{v}$};
\end{tikzpicture}

On peut vérifier que $\mathbf{v} = 2\mathbf{b}_1
+ 5 \mathbf{b}_2$

\begin{tikzpicture}
\clip (-2,-2) rectangle (10cm,6cm);
\pgftransformcm{1}{-0.5}{1}{1}{\pgfpoint{0cm}{0cm}};
\draw[style=help lines] (-14,-14) grid[step=1cm] (14,14);
\draw[thick,blue,->]  (0,0)--(2,0) node[anchor=south]{$2\mathbf{b}_1$};
\draw[thick,blue,->] (2,0)--(2,5) node[anchor=east]{$5\mathbf{b}_2$};
\draw[thick,black,->](0,0)--(2,5) node[anchor=south]{$\mathbf{v}$};
\end{tikzpicture}

\newpage
Superposons les deux systèmes de coordonnées:

\begin{tikzpicture}
\clip (-2,-2) rectangle (10cm,6cm);
\draw[style=help lines,dashed] (-14,-14) grid[step=1cm] (14,14);
\draw[thick,red,->]  (0,0)--(1,0) node[anchor=south]{$\mathbf{e}_1$};
\draw[thick,red,->] (0,0)--(0,1) node[anchor=east]{$\mathbf{e}_2$};
\draw[thick,black,->](0,0)--(7,4) node[anchor=south]{$\mathbf{v}$};
\pgftransformcm{1}{-0.5}{1}{1}{\pgfpoint{0cm}{0cm}};
\draw[style=help lines] (-14,-14) grid[step=1cm] (14,14);
\draw[thick,blue,->]  (0,0)--(1,0) node[anchor=south]{$\mathbf{b}_1$};
\draw[thick,blue,->] (0,0)--(0,1) node[anchor=east]{$\mathbf{b}_2$};
\end{tikzpicture}

Dans le plan cartésien, on écrit habituellement:
\[
\mathbf{e}_1 = \begin{bmatrix}
1 \\
0
\end{bmatrix}
\qquad
\mathbf{e}_2 = \begin{bmatrix}
0 \\
1
\end{bmatrix}
\qquad
\mathbf{v} = 7\mathbf{e}_1
+ 4 \mathbf{e}_2 = \begin{bmatrix}
7 \\ 4
\end{bmatrix}
\]
Dans ce système de coordonnées, défini par cette
base canonique, on peut vérifier que
\[
\mathbf{b}_1 = \begin{bmatrix}\hphantom{-}
1\\ -\frac12
\end{bmatrix}
\qquad
\mathbf{b}_2 = \begin{bmatrix}
1\\ 1
\end{bmatrix}
\qquad
\mathbf{v} = 2\mathbf{b}_1
+ 5 \mathbf{b}_2 = 2\begin{bmatrix}
\hphantom{-}1\\ -\frac12
\end{bmatrix} + 5 \begin{bmatrix}
1\\ 1
\end{bmatrix}
=
\begin{bmatrix}
\hphantom{-}2 + 5 \\ -1 + 5
\end{bmatrix}
= \begin{bmatrix}
7 \\ 4
\end{bmatrix}
\]
Notez que, utilisant la notation de la multiplication
des matrices qu'on a vu jusqu'à présent, on a également l'égalité suivante:
\[
2\begin{bmatrix}
\hphantom{-}1\\ -\frac12
\end{bmatrix} + 5 \begin{bmatrix}
1\\ 1
\end{bmatrix}
=
\begin{bmatrix}
\hphantom{-}1 & 1 \\
-\frac12 & 2
\end{bmatrix}
\begin{bmatrix}
2 \\ 5
\end{bmatrix}
\]
La multiplication d'une matrice par un vecteur
est égale à une combinaison linéaire des colonnes
de la matrice.

\newpage
Supposons que l'on veuille exprimer le vecteur $\mathbf{v}$ comme une matrice colonne, de taille $2\times 1$, \textbf{sans faire référence aux coordonnées du plan cartésien}, mais en utilisant la base
$B = \{\mathbf{b}_1,\mathbf{b}_2\}$

\begin{tikzpicture}
\clip (-2,-2) rectangle (10cm,6cm);
\pgftransformcm{1}{-0.5}{1}{1}{\pgfpoint{0cm}{0cm}};
\draw[style=help lines] (-14,-14) grid[step=1cm] (14,14);
\draw[thick,blue,->]  (0,0)--(2,0) node[anchor=south]{$2\mathbf{b}_1$};
\draw[thick,blue,->] (2,0)--(2,5) node[anchor=east]{$5\mathbf{b}_2$};
\draw[thick,black,->](0,0)--(2,5) node[anchor=south]{$\mathbf{v}$};
\end{tikzpicture}

\[
\mathbf{v} = 2\mathbf{b}_1 + 5 \mathbf{b}_2
\]
Ce qu'on peut faire est de définir les matrices
suivantes:
\[
\mathbf{b}_1 = \begin{bmatrix}
1 \\ 0
\end{bmatrix}_B
\qquad
\mathbf{b}_2 = \begin{bmatrix}
0 \\ 1
\end{bmatrix}_B
\qquad
\mathbf{v} = 2\mathbf{b}_1
+ 5 \mathbf{b}_2
=
\begin{bmatrix}
5 \\ 2
\end{bmatrix}_B
\]
où l'indice ${}_B$ fait référence à la base $B$.

On a une relation entre les coordonnées cartésiennes
et celles de la base $B$:
\[
2\begin{bmatrix}
\hphantom{-}1\\ -\frac12
\end{bmatrix} + 5 \begin{bmatrix}
1\\ 1
\end{bmatrix}
=
\begin{bmatrix}
\hphantom{-}1 & 1 \\
-\frac12 & 2
\end{bmatrix}
\begin{bmatrix}
2 \\ 5
\end{bmatrix}_B
=
P_B\begin{bmatrix}
2 \\ 5
\end{bmatrix}_B
\]
La matrice $P_B$ est appelée la matrice de passage de la base $B$ à la base cartésienne habituelle. Les colonnes
de $P_B$ sont les vecteurs $\mathbf{b}_1,\mathbf{b}_2$
exprimés dans les coordonnées cartésiennes.

De façon plus générale, supposons que l'on veuille obtenir les coordonnées d'un vecteur $v$ dans une base $C$, on aura:

\[
\begin{bmatrix}
v_1 \\ v_2
\end{bmatrix}_C
=
\begin{bmatrix}
(\mathbf{b}_1)_C
&
(\mathbf{b}_1)_C
\end{bmatrix}
\begin{bmatrix}
y_1 \\ y_2
\end{bmatrix}_B
=
\begin{bmatrix}
\begin{pmatrix}
b_{11} \\
b_{12}
\end{pmatrix}_C
&
\begin{pmatrix}
b_{21} \\
b_{22}
\end{pmatrix}_C
\end{bmatrix}
\begin{bmatrix}
v_1 \\ v_2
\end{bmatrix}_B
\]
On peut vérifier que
\[
\begin{bmatrix}
b_{11} \\ b_{12}
\end{bmatrix}_C
=
\begin{bmatrix}
\begin{pmatrix}
b_{11} \\
b_{12}
\end{pmatrix}_C
&
\begin{pmatrix}
b_{21} \\
b_{22}
\end{pmatrix}_C
\end{bmatrix}
\begin{bmatrix}
1 \\ 0
\end{bmatrix}_B
\]
et
\[
\begin{bmatrix}
b_{21} \\ b_{22}
\end{bmatrix}_C
=
\begin{bmatrix}
\begin{pmatrix}
b_{11} \\
b_{12}
\end{pmatrix}_C
&
\begin{pmatrix}
b_{21} \\
b_{22}
\end{pmatrix}_C
\end{bmatrix}
\begin{bmatrix}
0 \\ 1
\end{bmatrix}_B
\]
\section*{Notation}
On utilise parfois la notation suivante:
\[
[\mathbf{x}]_C = \underset{\scriptscriptstyle C \leftarrow B}
P
\quad
[\mathbf{x}]_B
\]
Pour désigner la matrice de passage de la base $B$ à la base $C$.

Dans les cas où la matrice de passage est inversible, on peut vérifier que
\[
\underset{\scriptscriptstyle C \leftarrow B}
P
=
\underset{\scriptscriptstyle B \leftarrow C}
P^{-1}
\]

Dans notre cours, on utilisera seulement la matrice de passage pour les transformations d'une base formée des vecteurs propres (si cette base existe) à la base
cartésienne. On la désignera simplement par $P$, et son inverse par $P^{-1}$ comme d'habitude.
\end{document}