\chapter{Nombre de solutions}

Si je vous demandais de trouver la solution du
système d'équations linéaires
\[
\left\{
\begin{matrix}
x &=& 3\\
2x &=& 6
\end{matrix}
\right.
\]
je suis convaincu que vous y parviendriez même si
vous n'avez pas maîtrisé les techniques de calcul
avancées qu'on utilise en algèbre linéaire.
Néanmoins, je vais procéder comme si vous n'étiez pas
capable de trouver la solution vous-même.

Tout d'abord, on écrit la matrice augmentée correspondant
à ce système.
\[
\left\{
\begin{matrix}
{\color{red}1}x &=& {\color{blue}3}\\
{\color{red}2}x &=& {\color{blue}6}
\end{matrix}
\right.
\qquad \Rightarrow\qquad
\begin{bmatrix}[r|r]
{\color{red}1} & {\color{blue}3}\\
{\color{red}2} & {\color{blue}6}
\end{bmatrix}
\]
Ensuite, on effectue des opérations élémentaires sur les
lignes pour obtenir une matrice équivalente mais qui a
une forme échelonnée réduite. On peut faire ceci en remplaçant
la deuxième ligne par celle-là même à laquelle on soustrait
deux fois la première ligne
\[
L_2 - 2L_1 \rightarrow L_2 \qquad \Rightarrow\qquad
\begin{bmatrix}[c|r]
\mbox{\large{\textcircled{\small $1$}}} & 3\\
0 & 0
\end{bmatrix}
\]
J'ai
encerclé le pivot présent dans cette matrice parce que ceci est souvent utile\sidenote{Mais ce n'est pas le cas ici.}. On remarque que tous les coefficients sur la dernière ligne sont des zéros.
\textbf{Qu'est-ce que ceci peut bien vouloir dire?}\sidenote{
Dans plusieurs manuels d'algèbre linéaire, parce qu'on considère des exemples beaucoup plus compliqués qu'ici, on voit une ligne nulle apparaître uniquement dans les
cas où il existe un nombre infini de solutions. Ce faisant, les étudiants
pensent automatiquement que la présence d'une ligne entièrement nulle signifie que le système a un nombre infini de solutions, ce qui n'est 
évidemment pas le cas ici.}

Pour savoir ce que cette ligne représente, il faut revenir à la forme
équivalente où on a des équations avec des inconnues.
\[
\left\{
\begin{matrix}
x &=& 3\\
0\,x &=& 0 &\qquad {\color{blue}\Leftarrow\quad\mbox{Notez cette équation.}}
\end{matrix}
\right.
\]
Puisque multiplier un nombre par zéro donne zéro comme résultat, on sait que l'équation $0\,x=0$
sera vraie peu importe la valeur de $x$.
Comme cette équation ne nous apprend rien sur
la solution recherchée, on peut l'oublier.
La seule équation qui reste est
\[
x = 3
\]
Ceci est la solution de ce système d'équations
linéaires, comme vous l'aviez sûrement déjà obtenu. On n'a qu'une valeur possible pour
la variable $x$: on dit qu'on a une solution unique.

\begin{NotDef}
Dans une matrice augmentée, une ligne où tous les coefficients sont nuls
correspond à une équation qui ne nous donne absolument aucune information sur
la valeur des inconnues.
\end{NotDef}

\section{Aucune solution}

Considérez le système d'équations linéaires suivant:
\[
\left\{
\begin{matrix}
x &=& 3\\
x &=& 4
\end{matrix}
\right.
\]
Si $x=3$, la deuxième équation est fausse; si $x=4$, c'est la première équation qui est fausse.
Comme il est impossible de trouver une valeur
de $x$ qui fasse en sorte que les deux équations sont vraies, on dit que le système est \textbf{incompatible} ou \textbf{inconsistant}: il n'existe aucune solution. Voyons à quoi ceci
ressemble si on utilise la notation des
matrices augmentées.

La matrice augmentée équivalente à ce système d'équations linéaires est
\[
\left\{
\begin{matrix}
{\color{red}1}x &=& {\color{blue}3}\\
{\color{red}1}x &=& {\color{blue}4}
\end{matrix}
\right.
\qquad \Rightarrow\qquad
\begin{bmatrix}[r|r]
{\color{red}1} & {\color{blue}3}\\
{\color{red}1} & {\color{blue}4}
\end{bmatrix}
\]

On peut obtenir une matrice équivalente ayant
une forme échelonnée\sidenote{Mais pas réduite.} à l'aide d'une seule opération
élémentaire sur les lignes

\[
L_2 - L_1 \rightarrow L_2 \qquad \Rightarrow\qquad
\begin{bmatrix}[c|c]
\mbox{\large{\textcircled{\small $1$}}} & 3\\
0 & \mbox{\large{\textcircled{\small $1$}}}
\end{bmatrix}
\]
On voit qu'on a un pivot du côté droit de la barre verticale\sidenote{Si on faisant comme certains auteurs et qu'on n'incluait pas
une telle barre verticale, il serait encore plus
difficile d'observer qu'on a une situation
inhabituelle.}. \textbf{Qu'est-ce que cela
veut dire?}

Pour le savoir, on écrit l'équation correspondant à cette deuxième ligne de la matrice augmentée:
\[
[\quad{\color{red}0}\quad|\quad{\color{blue}1}\quad] \qquad \Rightarrow\qquad {\color{red}0}\,x = {\color{blue}1}
\]
On voit tout de suite qu'il n'y a aucune valeur
possible de $x$ qui fasse en sorte que cette équation soit vraie: le système n'a aucune solution.

\begin{NotDef}
Dans certains manuels, on vous suggère de mémoriser le fait que d'avoir un pivot dans la dernière colonne d'une matrice augmentée correspond à n'avoir aucune solution. \textbf{Ne faites pas ceci!} Plutôt, habituez-vous à passer de la forme d'une matrice augmentée à celle d'un système d'équations dans sa forme habituelle avec les inconnues indiquées lorsque vous pensez
qu'il pourrait ne pas y avoir de solutions.
\end{NotDef}

\section{Infinité de solutions}

Considérez le système d'équations suivant qui
n'a qu'une seule équation:
\[
 x + y = 1
\]
Ceci est l'équation d'une droite\sidenote{Dans le plan $x\text{---}y$.}. Il existe un nombre infini de
valeurs possibles pour $x$ et $y$ qui font en sorte que cette équation soit vérifiée.  Par exemple, on a $(0, 1), (1, 0), (-1, 2), (2, -1), \ldots$. 
On a donc un nombre infini de solutions.  
Dans le prochain chapitre, on verra une façon utile de dénoter toutes ces solutions.

\section{Résumé du chapitre}

\begin{NotProof}
Un système d'équations linéaires a soit:
\begin{itemize}
\item[$\quad\scriptstyle\bullet$] aucune solution;
\item[$\quad\scriptstyle\bullet$] une seule solution;
\item[$\quad\scriptstyle\bullet$] un nombre infini de solutions.
\end{itemize}
\leavevmode %% prevents LaTeX error
\end{NotProof}

\begin{Example}
Dans un problème où on vous demande de déterminer la ou les valeurs de 2 inconnues, $x$ et $y$,
en fonction d'une constante $k$ non spécifiée,
vous obtenez la matrice augmentée suivante:
\[
\begin{bmatrix}[cc|c]
1 \qquad& 2 & 3 \\
0 \qquad& k^2 - 4 & k-2
\end{bmatrix}
\]
La dernière ligne de cette matrice augmentée
correspond à l'équation suivante\sidenote{Dans le passé, à chaque fois que je donnais un problème semblable mais légèrement plus compliqué, des étudiants écrivent que ceci correspond plutôt à l'équation $$k^2 - 4 = k-2$$ C'est en partie
ce qui m'a motivé à écrire ce livre avec des exemples les plus simples possibles.}:
\[
(k^2 - 4)y = k-2
\]
Il y a trois cas possibles:
\begin{enumerate}
\item Si $k=2$, cette équation correspond à
$$0\,y = 0$$ qui est vrai peu importe la valeur de $y$. Cette équation ne nous donne aucune information sur les solutions possibles. On doit donc examiner l'équation qui reste, c'est-à-dire l'équation correspondant à la première ligne de la matrice augmentée; celle-ci est
\[
x + 2y = 3
\]
Ceci est l'équation d'une droite: il y a donc une
infinité de solutions dans ce cas.
\item Si $k=-2$, cette équation correspond à
$$0\,y = -4$$ Ceci est impossible peu importe la valeur de $y$: il n'y a donc aucune solution possible dans ce cas.
\item Pour les valeurs de $k$ différente de $2$ et de $-2$, on peut diviser de chaque côté par $k^2 -4$ pour trouver
\[
y = \frac{1}{k+2}
\]
Il y a donc une valeur unique pour $y$ et, après un peu de calculs supplémentaires, on peut vérifier qu'on obtient
\[
x = \frac{3k+4}{k+2}
\]
On a donc une solution unique dans ce dernier cas.  Par exemple, si $k=0$, la solution est $(x, y) = (2, \frac12)$.
\end{enumerate}
\leavevmode %%% prevents LaTeX error
\end{Example}