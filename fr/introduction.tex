\chapter{Introduction}
Le cours d'introduction à l'algèbre linéaire est souvent
le premier cours au niveau universitaire où les étudiants ont à apprendre des concepts de mathématiques avec une approche
typique de celle utilisée par les mathématiciens professionnels, basée sur l'abstraction et les démonstrations.  Ceci peut être un peu intimidant.

Selon Wikipédia\sidenote{\url{https://fr.wikipedia.org/wiki/Mathématiques}}, les mathématiques sont définies ainsi:

\begin{TrueDef}{Mathématiques}
Les mathématiques (ou la mathématique) sont un ensemble de connaissances abstraites résultant de raisonnements logiques appliqués à des objets divers tels que les nombres, les formes, les structures et les transformations. Elles sont aussi le domaine de recherche développant ces connaissances, ainsi que la discipline qui les enseigne.

Elles possèdent plusieurs branches telles que : l'arithmétique, l'algèbre, l'analyse, la géométrie, la logique mathématique, etc. Il existe également une certaine séparation entre les mathématiques pures et les mathématiques appliquées.
\end{TrueDef}

Dans ce livre, nous présentons les définitions formelles
dans des encadrés en bleu tel que celui ci-dessus. 
Un des buts de ce manuel est de vous aider à développer
une compréhension intuitive de certaines définitions
que l'on retrouve en algèbre linéaire.
Pour vous aider, nous allons souvent utiliser en premier
des définitions un peu plus informelles.  Par exemple, votre expérience dans vos cours de mathématiques vous fait
peut-être penser que la définition des mathématiques ressemble plutôt à la suivante:

\begin{NotDef}
Les mathématiques qu'on voit au niveau universitaire
ont été inventés par des sadiques qui inventent des
termes compliqués\sidenote{Comme le "non-sens abstrait" \url{https://fr.wikipedia.org/wiki/Abstract_nonsense}} uniquement dans le but de décourager les étudiants.
\end{NotDef}



\section{Au sujet des démonstrations}

Les mathématiciens \textbf{adorent} les démonstrations\sidenote{Certains utilisent parfois le mot \textit{preuve} au lieu de démonstration, ce qui est un anglicisme.}.
Le but de ce manuel est de vous donner une idée intuitive
des concepts de base de l'algèbre linéaire en évitant les démonstrations. Cependant, nous utilisons
généralement des exemples simples pour guider votre intuition\sidenote{Nous incluons parfois des énoncés qui peuvent être faux dans un contexte différent.}.

\begin{NotProof}
Un exemple n'est pas une démonstration. Un argument de plausibilité n'est pas une démonstration. Une démonstration
est une suite précise de propositions logiques. Le manuel recommandé par votre professeur contient toutes les démonstrations requises.
Votre professeur inclut des démonstrations à faire dans les
examens uniquement\sidenote{Ceci n'est pas toujours vrai; 
ça dépend du professeur qui enseigne le cours} dans le but de faire baisser votre note dans le cours.
\end{NotProof}

Les notes en marge\sidenote{Comme celle-ci.} ajoutent des
détails que vous trouverez parfois utile\sidenote{Du moins, je l'espère.}.

\section{Au sujet des exemples}

En général, je cherche à choisir des exemples tellement simples qu'ils sont probablement jugés par votre professeur comme l'étant beaucoup trop pour être utilisé en classe,
ou dans votre manuel de cours. Il est vrai que, très souvent,
les exemples que j'ai choisis peuvent cacher la complexité
d'un sujet donné: vous devriez les considérer comme un
simple premier pas dans votre voyage d'apprentissage du sujet.
