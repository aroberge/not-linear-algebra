\chapter{Matrices}
Sans avoir défini ce qu'était une \textbf{matrice}, nous avons
néanmoins introduit un objet appelé \textbf{matrice augmentée}.
Sans plus attendre, voici une définition formelle d'une \textbf{matrice}.

\begin{TrueDef}{Matrice}
Soit $m$ et $n$ deux entier positifs; une matrice de \textbf{taille}
$m\times n$
est une collection de $mn$ nombres, appelés coefficients, arrangés dans un tableau rectangulaire:
\[
\begin{matrix}[cc]
&\text{$n$ colonnes} \\
\text{$m$ lignes}& \begin{pmatrix}
        a_{11} & a_{12} & \ldots & a_{1n}\\
        a_{21} & a_{22} & \ldots & a_{2n}\\
        \vdots & \vdots & \vdots & \vdots \\
        a_{m1} & a_{m2} & \ldots & a_{mn}
        \end{pmatrix}
  \end{matrix}
\]
\end{TrueDef}

Vous avez peut-être remarqué\sidenote{
Croyez-le ou non, ceci n'est pas une coïncidence.} que la notation utilisée pour
les coefficients dans la définition
d'une matrice ressemble étrangement à celle utilisée dans la définition d'un système d'équations linéaires.

\begin{TrueDef}{Système d'équations linéaires}
Un système de d'équations linéaires est une
collection de $m$ équations linéaires qui portent
sur les mêmes $n$ inconnues:
\[
\left\{
	\begin{matrix}
	a_{11}x_1 &+& a_{12}x_2 &+& \ldots &+& a_{1n}x_n &=& b_1 \\
	a_{21}x_1 &+& a_{22}x_2 &+& \ldots &+& a_{2n}x_n &=& b_2 \\
	\vdots && \vdots &&  && \vdots && \vdots \\
	a_{m1}x_1 &+& a_{m2}x_2 &+& \ldots &+& a_{mn}x_n &=& b_m
	\end{matrix}
	\right.
\]
\end{TrueDef}

\begin{Example}
Voici une matrice de taille $2 \times 3$:
\[\begin{bmatrix} 2 & 1 & 0 \\ 1 & 3 & 5 \\ \end{bmatrix}\]

Au lieu d'utiliser des crochets, $[\ldots]$,
  on utilise parfois des parenthèses pour encadrer les coefficients d'une matrice:
  \[\begin{pmatrix} 2 & 1 & 0 \\ 1 & 3 & 5 \\ \end{pmatrix}\]
\end{Example}

Nous allons étudier plusieurs propriétés des matrices dans ce
livre. Pour l'instant, nous allons nous limiter à définir deux
\textit{formes} que peuvent prendre les matrices. 


\section{Matrice de forme échelonnée}

%%%%%%%%%%%%%%%%%%%%%%%%

La matrice suivante est dans la \textbf{forme échelonnée}:
\[
\begin{bmatrix}[ccccccccc]
p_1 & *  & * & * & * & * & * & * & * \\
0 & 0 & p_2 & * & * & * & * & * & * \\
0 & 0 & 0 & p_3 & * & * & * & * & * \\ 
0 & 0 & 0 & 0 & 0 & 0 & p_4 & * & * \\ 
0 & 0 & 0 & 0 & 0 & 0 & 0 & 0 & p_5 
\end{bmatrix}
\]
Dans cette matrice, les coefficients $p_1, p_2, \ldots$
sont tous différents de zéro. De plus, les coefficients
identifiés par un astérisque, $*$, peuvent prendre n'importe
quelle valeur. 
Une matrice est en forme échelonnée si
le premier coefficient non-nul d'une ligne donnée est toujours à 	la droite du coefficient non-nul de la ligne précédente. On appelle
ce coefficient non-nul un \textbf{pivot}. Pour aider à identifier
ce qu'on entent par une forme échelonnée, nous écrivons la matrice
à nouveau mais, cette fois, nous 
identifions une forme en escalier\sidenote{C'est à cette
forme en escalier que le mot \textit{échelonné} fait
référence.} sous laquelle tous
les coefficients doivent être nuls.
\[
\newcommand*{\temp}{\multicolumn{1}{c|}{\color{red}0}}
\begin{bmatrix}[ccccccccc]
p_1 & * & *  & * & * & * & * & * & * \\ \cline{1-2}
\color{red}0 & \temp & p_2 & * & * & * & * & * & *  \\ \cline{3-3}
\color{red}0 &\color{red}0 & \temp & p_3 & * & * & * & * & * \\ \cline{4-6}
\color{red}0 &\color{red}0 & \color{red}0 & \color{red}0 & \color{red}0 & \temp & p_4 & * & * \\ \cline{7-8}
\color{red}0 &\color{red}0 & \color{red}0 &\color{red}0 & \color{red}0 & \color{red}0 & \color{red}0 & \temp & p_5 
\end{bmatrix}
\]
Une matrice est dans la forme échelonnée \textbf{réduite} si,
en plus d'être dans une forme échelonnée, tous les pivots
sont égaux à 1, et tous les autres coefficients dans un colonne
où il y a un pivot, sont égaux à 0.

\[
\newcommand*{\tempr}{\multicolumn{1}{c|}{\color{red}0}}
\newcommand*{\tempb}{\multicolumn{1}{c|}{\color{blue}0}}
\begin{bmatrix}[ccccccccc]
\mbox{\large{\textcircled{\small $1$}}} & * & \color{blue}0 & \color{blue}0 & * & * & \color{blue}0 & * & \color{blue}0 \\ \cline{1-2}
\color{blue}0 &\tempr & \mbox{\large{\textcircled{\small $1$}}} & \color{blue}0 & * & * & \color{blue}0 & * & \color{blue}0 \\ \cline{3-3}
\color{blue}0 &\color{red}0 &  \tempb & \mbox{\large{\textcircled{\small $1$}}} & * & * & \color{blue}0 & * & \color{blue}0 \\ \cline{4-6}
\color{blue}0 &\color{red}0 & \color{blue}0 & \color{blue}0 & \color{red}0 & \tempr & \mbox{\large{\textcircled{\small $1$}}} & * & \color{blue}0 \\ \cline{7-8}
\color{blue}0 &\color{red}0 & \color{blue}0 & \color{blue}0 & \color{red}0 & \color{red}0 & \color{blue}0 & \tempr & \mbox{\large{\textcircled{\small $1$}}} 
\end{bmatrix}
\]
Dans la matrice ci-dessus, on a encerclé les pivots et écrits
tous les zéros qui sont dans une \textbf{colonne pivot}\sidenote{
Une \textbf{colonne pivot} est une colonne où on retrouve un pivot.}
en bleu.

\begin{Example}
La matrice augmentée
\[
\begin{bmatrix}[ccr|r]
\mbox{\large{\textcircled{\small 1}}} & 0 & 0 & 2\\
0 & \mbox{\large{\textcircled{\small 1}}} & 4 & 3\\
0 & 0 & 0 & 0
\end{bmatrix}
\]
est dans une forme échelonnée réduite, où on a
identifié les pivots en les encerclant.
\end{Example}